\documentclass[11pt]{article}

%% import packages to avoid too many imports on main file
\usepackage{amsmath}
\usepackage{amssymb}
\usepackage{amsthm}
\usepackage{enumerate}
\usepackage{enumitem}
\usepackage{float}
\usepackage{graphicx}
\usepackage{hyperref}
\usepackage{ifthen}
\usepackage{lipsum}
\usepackage{mdframed}
\usepackage{tcolorbox}
\usepackage{xcolor}
\usepackage[sfdefault]{plex-sans}
\usepackage{tikz}

\tcbuselibrary{breakable}

\newcommand{\R}{\ensuremath{\mathbb{R}}}
\newcommand{\N}{\ensuremath{\mathbb{N}\hspace{1.0mm}}}
\newcommand{\Z}{\ensuremath{\mathbb{Z}}}
\newcommand{\Q}{\ensuremath{\mathbb{Q}}}
\newcommand{\C}{\ensuremath{\mathbb{C}}}

% formatting
\newcommand{\hs}{\hspace{1.0mm}} % horizontal space; mostly in equations
\makeatletter
\def\exam@MakeFramed#1{\begin{mdframed}}
\def\endexam@MakeFramed{\end{mdframed}}
\makeatother


\newtcolorbox{customsolutionbox}[1][]{
    colback=white,
    colframe=gray!50!black,
    sharp corners=southwest,
    rounded corners=southeast,
    title=Solution,
    breakable,
    #1 % answer
}

% \newcommand{\customequation}[1]{
%     \vspace{5.0mm}
%     \begin{center}
%         \begin{align*}
%             #1
%         \end{align*}
%     \end{center}
%     % \vspace{5.0mm}
% }

%% credit statement
\newcommand{\creditstatement}[1][alone]{
    % \ifthenelse{\equal{#1}{alone}}{
    \begin{center}
        {\fontsize{14}{16}\selectfont \textbf{\textit{Credit statement: I worked alone on this assignment with reference to: #1} \\}}
    \end{center}
    % }{
    % \begin{center}
    %     {\fontsize{14}{16}\selectfont\textbf{\textit{Credit statement: I discussed the problems with #1, but worked alone on my solutions} \\}}
    % \end{center}
    % }
}

%% heading
\newcommand{\customheading}[4]{ % #1=Date, #2=Title
    \fbox{%
        \begin{minipage}{\textwidth}
        \begin{tikzpicture}[overlay, remember picture]
            \node[anchor=north west, font=\bfseries\itshape] at (0,0) {Abdibaset Bare};
            \node[anchor=north east, font=\bfseries] at (\textwidth,0) {#1};
            \node[anchor=north, font=\Large\bfseries\itshape] at (\textwidth/2,-1.1) {#2};
            \node[anchor=south west, font=\itshape] at (0,-3) {#3};
            \node[anchor=south east, font=\itshape] at (\textwidth,-3) {#4};
        \end{tikzpicture}
        \vspace{3cm}
        \end{minipage}%
    }
}

\newcommand{\customfigure}[3]{
    \begin{figure}[H]
        \centering
        \includegraphics[width=25mm]{#1}
        \caption{#2}
        \label{fig:#3}
    \end{figure}
}

%% definitions
\newtheorem{theorem}{Theorem}
\newtheorem*{definition}{Definition}
\begin{document}
\customheading{\today}{Sets: Language in set systems}{Fall 2024}{CS30 - Discrete Maths}

\section*{Symbols}
\begin{enumerate}
    \item Subsets
        \begin{enumerate}[label=\alph*]
            \item $\subset$: Example: $A \subset B$ means A is subset of B, but may or may not be equal to B.
            \item $\subseteq$: Example $A \subseteq B$ means A is a subset of B, and can be equal to B. All the elements of A are in B. \textit{possibility of being equal}.
            \item $\subsetneq$: Means strict subset; that's all elements of A are in B.
        \end{enumerate}
    \item Union($\cup$): distint elements in subsets in one set.
    \item Intersection($\cap$): common elements in different sets.
    \item Difference ($\setminus$): removing an element from a set.
\end{enumerate}
\section*{Definition of functions}
\begin{gather}
    F : \{1, 2, 3, \ldots, n\} \times \Z \rightarrow \{0, 1\}
\end{gather}

This means $F$ takes inputs from a \textit{finite set} of integers $\{1, 2, 3 \ldots, n\}$ and set of all integers, $\Z$, to give an output $\{0, 1\}$

Purpose:
\begin{gather}
    F(i, j) =
    \begin{cases}
        1 & \text{if } \exists S \subseteq \{1, 2, 3, \ldots, j\} \hs \text{such that} \sum_{i \in S} a_i = k \\
        0 & \text{otherwise}
    \end{cases}
\end{gather}

\textbf{In english}: $F(j, k) = 1$ if \textit{there exists} a subset $S$ of the finite set of integers $\{1, 2, 3, \ldots j\}$ such that the sum of all its elements equals $k$; otherwise $F(j, k) = 0$. \\

The subset $S$ is composed of indices from a finite set, and each index $i \in S$ corresponds to a value $a_i$

\begin{enumerate}
    \item if $n=5$ and $(a_{1}, a_{2},  a_{3}, a_{4}, a_{5}) = (1, 2, 4, 5, 8)$, then write down the following values with one line justification. This part is to make sure you understand the definitions
    \begin{itemize}
        \item What is $F(j, 0)$ for $1 \leq j \leq 5$?
        \begin{gather}
            F(1, 0) = 1 \\ F(2, 0) = 1 \\ F(3, 0) = 1 \\ F(4, 0) = 1 \\ F(5, 0) = 1 \\
            \text{This means that} \hs  S = \emptyset
        \end{gather}


        \item What is $F(0,0)$?
            \begin{gather}
                F(0,0) = 0 \\ \text{Means} \hs S = \emptyset
            \end{gather}
        \item What is $F(3, 3)$? What $F(3, 8)$?
        \begin{gather}
            F(3, 3) = 0 \hs \text{because there exists a  subset} (1, 2) \hs \text{such that} \hs \sum (1, 2) = 3 \\
            F(3, 8) = 0 \hs \text{The largest subset is} (1, 2, 4) whose \sum (1, 2, 3) = 7
        \end{gather}
        Since $8 > 7$, there can't be any subset of the first 3 elements that adds up to 8.
    \end{itemize}
\end{enumerate}
\end{document}