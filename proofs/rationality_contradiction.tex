\documentclass[11pt]{article}

%% import packages to avoid too many imports on main file
\usepackage{amsmath}
\usepackage{amssymb}
\usepackage{amsthm}
\usepackage{enumerate}
\usepackage{enumitem}
\usepackage{float}
\usepackage{graphicx}
\usepackage{hyperref}
\usepackage{ifthen}
\usepackage{lipsum}
\usepackage{mdframed}
\usepackage{tcolorbox}
\usepackage{xcolor}
\usepackage[sfdefault]{plex-sans}
\usepackage{tikz}

\tcbuselibrary{breakable}

\newcommand{\R}{\ensuremath{\mathbb{R}}}
\newcommand{\N}{\ensuremath{\mathbb{N}\hspace{1.0mm}}}
\newcommand{\Z}{\ensuremath{\mathbb{Z}}}
\newcommand{\Q}{\ensuremath{\mathbb{Q}}}
\newcommand{\C}{\ensuremath{\mathbb{C}}}

% formatting
\newcommand{\hs}{\hspace{1.0mm}} % horizontal space; mostly in equations
\makeatletter
\def\exam@MakeFramed#1{\begin{mdframed}}
\def\endexam@MakeFramed{\end{mdframed}}
\makeatother


\newtcolorbox{customsolutionbox}[1][]{
    colback=white,
    colframe=gray!50!black,
    sharp corners=southwest,
    rounded corners=southeast,
    title=Solution,
    breakable,
    #1 % answer
}

% \newcommand{\customequation}[1]{
%     \vspace{5.0mm}
%     \begin{center}
%         \begin{align*}
%             #1
%         \end{align*}
%     \end{center}
%     % \vspace{5.0mm}
% }

%% credit statement
\newcommand{\creditstatement}[1][alone]{
    % \ifthenelse{\equal{#1}{alone}}{
    \begin{center}
        {\fontsize{14}{16}\selectfont \textbf{\textit{Credit statement: I worked alone on this assignment with reference to: #1} \\}}
    \end{center}
    % }{
    % \begin{center}
    %     {\fontsize{14}{16}\selectfont\textbf{\textit{Credit statement: I discussed the problems with #1, but worked alone on my solutions} \\}}
    % \end{center}
    % }
}

%% heading
\newcommand{\customheading}[4]{ % #1=Date, #2=Title
    \fbox{%
        \begin{minipage}{\textwidth}
        \begin{tikzpicture}[overlay, remember picture]
            \node[anchor=north west, font=\bfseries\itshape] at (0,0) {Abdibaset Bare};
            \node[anchor=north east, font=\bfseries] at (\textwidth,0) {#1};
            \node[anchor=north, font=\Large\bfseries\itshape] at (\textwidth/2,-1.1) {#2};
            \node[anchor=south west, font=\itshape] at (0,-3) {#3};
            \node[anchor=south east, font=\itshape] at (\textwidth,-3) {#4};
        \end{tikzpicture}
        \vspace{3cm}
        \end{minipage}%
    }
}

\newcommand{\customfigure}[3]{
    \begin{figure}[H]
        \centering
        \includegraphics[width=25mm]{#1}
        \caption{#2}
        \label{fig:#3}
    \end{figure}
}

%% definitions
\newtheorem{theorem}{Theorem}
\newtheorem*{definition}{Definition}
\begin{document}
\customheading{\today}{Proof by contradtion: Rationality of Numbers}{Fall 2024}{CS30 - Discrete Maths}

\section*{}
\begin{enumerate}
    \item  Prove that $log_{2}3$ is irrational.
    \begin{proof}
        Suppose $log_{2}3$ were rational, i.e. $log_{2}3 = \frac{p}{3}$ where $p$ and $q$ are some positive integers. Then $3 = 2^{\frac{p}{q}} \equiv 3^{q} = 2^{p}$. An odd number cannot be equal to an even number hence $log_{2}3$ is irrational. \\

        Since $log_{2}3$ is an irrational number, $2log_{2}3$ is irrational.
        But it is not always the case. For example, let $x = \sqrt{2}$ and $y=log_{2}3$ then
        \begin{equation}
            x^{y} = (\sqrt{2})^{2log_{2}3} = \sqrt{2^{2}}^{log_{2}3} = 2^{log_{2}3} = 3
        \end{equation}
    \end{proof}

    \item Prove that $\sqrt{6}$ is irrational.
    \begin{proof}
        Suppose $\sqrt{6}$ were rational, i.e. $\sqrt{6} = \frac{p}{q}$ where $p$ and $q$ are some positive integers, and at least one of them is odd.If you square all sides, you will have, $ 6 = (\frac{p}{q})^{2} \equiv 6q^{2} = p^{2} = 2\cdot3q^{2} = p^{2}$. Therefore $p^{2}$ must be even, then $p$ is even. Let's say $p=2k \Rightarrow p^{2} = 4k^{2} = 2\cdot 3q^{2} \equiv \frac{2}{3}k^{2} = q^{2}$, but $q^{2}$ is a multiple of $2$ hence $q^{2}$ is even, and so is $q$. This is a contradtion because we assumed that either $p$ or $q$ is odd.
    \end{proof}
\end{enumerate}

\end{document}