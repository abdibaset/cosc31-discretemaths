\documentclass[11pt]{exam}

%% import packages to avoid too many imports on main file
\usepackage{amsmath}
\usepackage{amssymb}
\usepackage{amsthm}
\usepackage{enumerate}
\usepackage{enumitem}
\usepackage{float}
\usepackage{graphicx}
\usepackage{hyperref}
\usepackage{ifthen}
\usepackage{mdframed}
\usepackage{tcolorbox}
\usepackage{xcolor}
\usepackage[sfdefault]{plex-sans}
\usepackage{tikz}


\newcommand{\R}{\ensuremath{\mathbb{R}}}
\newcommand{\N}{\ensuremath{\mathbb{N}\hspace{1.0mm}}}
\newcommand{\Z}{\ensuremath{\mathbb{Z}}}
\newcommand{\Q}{\ensuremath{\mathbb{Q}}}
\newcommand{\C}{\ensuremath{\mathbb{C}}}

% formatting
\newcommand{\hs}{\hspace{1.0mm}} % horizontal space; mostly in equations
\makeatletter
\def\exam@MakeFramed#1{\begin{mdframed}}
\def\endexam@MakeFramed{\end{mdframed}}
\makeatother


\newtcolorbox{customsolutionbox}[1][]{
    colback=white,
    colframe=gray!50!black,
    sharp corners=southwest,
    rounded corners=southeast,
    title=Solution,
    #1 % answer
}

% \newcommand{\customequation}[1]{
%     \vspace{5.0mm}
%     \begin{center}
%         \begin{align*}
%             #1
%         \end{align*}
%     \end{center}
%     % \vspace{5.0mm}
% }

%% credit statement
\newcommand{\creditstatement}[1][alone]{
    % \ifthenelse{\equal{#1}{alone}}{
    \begin{center}
        {\fontsize{14}{16}\selectfont \textbf{\textit{Credit statement: I worked alone on this assignment with reference to: #1} \\}}
    \end{center}
    % }{
    % \begin{center}
    %     {\fontsize{14}{16}\selectfont\textbf{\textit{Credit statement: I discussed the problems with #1, but worked alone on my solutions} \\}}
    % \end{center}
    % }
}

%% heading
\newcommand{\customheading}[4]{ % #1=Date, #2=Title
    \fbox{%
        \begin{minipage}{\textwidth}
        \begin{tikzpicture}[overlay, remember picture]
            \node[anchor=north west, font=\bfseries\itshape] at (0,0) {Abdibaset Bare};
            \node[anchor=north east, font=\bfseries] at (\textwidth,0) {#1};
            \node[anchor=north, font=\Large\bfseries\itshape] at (\textwidth/2,-1.1) {#2};
            \node[anchor=south west, font=\itshape] at (0,-3) {#3};
            \node[anchor=south east, font=\itshape] at (\textwidth,-3) {#4};
        \end{tikzpicture}
        \vspace{3cm}
        \end{minipage}%
    }
}

\newcommand{\customfigure}[3]{
    \begin{figure}[H]
        \centering
        \includegraphics[width=25mm]{#1}
        \caption{#2}
        \label{fig:#3}
    \end{figure}
}

%% definitions
\newtheorem{theorem}{Theorem}
\newtheorem*{definition}{Definition}
\begin{document}
\customheading{\today}{Problem 11bc: Proof by Contradiction}{Fall 2024}{CS30 - Discrete Maths}
\begin{center}
    Credit statement: \textit{ I have worked alone on this assignment with reference to this resource \href{https://en.wikipedia.org/wiki/Fundamental_theorem_of_arithmetic}{wikipedia}}
\end{center}


\section*{}
\begin{parts}
    \part Let $N$ be any natural number at most $2^{k}$ for some $k \in \N$. Prove that $N$ can have at most $k$ distinct prime factors.
    \begin{customsolutionbox}
        In predicate logic, this statement and its negation can be presented as:
        \begin{gather}
            \forall N \in \N, \exists k \in \N : (N \leq 2^{k}) \Rightarrow \text{number of distinct prime number of} \hs N \leq k\\
            \text{negation}: \exists N \in \N, \forall k \in \N: (N \leq 2^{k}) \Rightarrow \text{number of distinct prime number of} \hs N > k
        \end{gather}

        \begin{proof}
            Suppose there exists a natural number $N$, and any natural number $k$ such that $N \leq 2^{k}$, and $N$ has more than $k$ distinct prime numbers.\\

            My supposition is that $N$ has at least $k+1$ prime numbers i.e., $\{p_{1}, p_{2}, \ldots, p_{k+1}\}$ for \textit{every} $k$. Since the smallest prime is $2$ we cay that $p_{1}\cdot p_{2} \cdot \ldots p_{k+1} \geq 2^{k+1}$, thus we have $N \geq  p_{1}\cdot p_{2} \cdot \ldots p_{k+1} \geq 2^{k+1}$ because $N$ is atleast the product of its prime numbers. However, this contradicts my initial assumption that $N \leq 2^{k}$. Therefore, assumption $N$ has at most $k$ prime numbers for some $k$ is true.\\
        \end{proof}
    \end{customsolutionbox}

      \part Prove that there can be no integers $x$ and $y$ such that $4x^{2} = y^{2} +1$
      \begin{customsolutionbox}
        Statement and its negation \\

        \begin{gather}
            \forall x \in \Z, \forall y  \in \Z: 4x^{2}  \not = y^{2} + 1 \\
            \text{negation }: \exists x \in \Z, \exists y \in \Z: 4x^{2}  = y^{2} + 1
        \end{gather}
        Suppose there is some integer $x$ and $y$ such that $4x^{2} = y^{2} + 1$. \\
        Rearranging:
        \begin{gather}
            4x^{2} - y^{2} = 1 \\
            (2x - y)\cdot(2x + y) = 1 \hs \text{difference of squares} \\
        \end{gather}
        let $p = 2x-y$ and $q = 2x+y$.
        \begin{gather}
            p = \frac{1}{2x+y}  \\
            q = \frac{1}{2x-y} \\
            p \cdot q = 1
        \end{gather}

        As shown, $p$ is not an integer  hence either $x$ or $y$ or both are not integers because the difference of two integers is an integer. Similarly for $q$ is not an integer hence either $x$ or $y$ or both are not integers because sum of two integers is an integer. As observed, either $x$ or $y$ or both are not integers. Therefore, the contradiction that there are some integers $x$ and $y$ such that $4x^{2}  = y^{2} + 1$
      \end{customsolutionbox}
\end{parts}

\end{document}