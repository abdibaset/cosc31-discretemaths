\documentclass[11pt]{exam}

%% import packages to avoid too many imports on main file
\usepackage{amsmath}
\usepackage{amssymb}
\usepackage{amsthm}
\usepackage{enumerate}
\usepackage{enumitem}
\usepackage{float}
\usepackage{graphicx}
\usepackage{hyperref}
\usepackage{ifthen}
\usepackage{lipsum}
\usepackage{mdframed}
\usepackage{tcolorbox}
\usepackage{xcolor}
\usepackage[sfdefault]{plex-sans}
\usepackage{tikz}

\tcbuselibrary{breakable}

\newcommand{\R}{\ensuremath{\mathbb{R}}}
\newcommand{\N}{\ensuremath{\mathbb{N}\hspace{1.0mm}}}
\newcommand{\Z}{\ensuremath{\mathbb{Z}}}
\newcommand{\Q}{\ensuremath{\mathbb{Q}}}
\newcommand{\C}{\ensuremath{\mathbb{C}}}

% formatting
\newcommand{\hs}{\hspace{1.0mm}} % horizontal space; mostly in equations
\makeatletter
\def\exam@MakeFramed#1{\begin{mdframed}}
\def\endexam@MakeFramed{\end{mdframed}}
\makeatother


\newtcolorbox{customsolutionbox}[1][]{
    colback=white,
    colframe=gray!50!black,
    sharp corners=southwest,
    rounded corners=southeast,
    title=Solution,
    breakable,
    #1 % answer
}

% \newcommand{\customequation}[1]{
%     \vspace{5.0mm}
%     \begin{center}
%         \begin{align*}
%             #1
%         \end{align*}
%     \end{center}
%     % \vspace{5.0mm}
% }

%% credit statement
\newcommand{\creditstatement}[1][alone]{
    % \ifthenelse{\equal{#1}{alone}}{
    \begin{center}
        {\fontsize{14}{16}\selectfont \textbf{\textit{Credit statement: I worked alone on this assignment with reference to: #1} \\}}
    \end{center}
    % }{
    % \begin{center}
    %     {\fontsize{14}{16}\selectfont\textbf{\textit{Credit statement: I discussed the problems with #1, but worked alone on my solutions} \\}}
    % \end{center}
    % }
}

%% heading
\newcommand{\customheading}[4]{ % #1=Date, #2=Title
    \fbox{%
        \begin{minipage}{\textwidth}
        \begin{tikzpicture}[overlay, remember picture]
            \node[anchor=north west, font=\bfseries\itshape] at (0,0) {Abdibaset Bare};
            \node[anchor=north east, font=\bfseries] at (\textwidth,0) {#1};
            \node[anchor=north, font=\Large\bfseries\itshape] at (\textwidth/2,-1.1) {#2};
            \node[anchor=south west, font=\itshape] at (0,-3) {#3};
            \node[anchor=south east, font=\itshape] at (\textwidth,-3) {#4};
        \end{tikzpicture}
        \vspace{3cm}
        \end{minipage}%
    }
}

\newcommand{\customfigure}[3]{
    \begin{figure}[H]
        \centering
        \includegraphics[width=25mm]{#1}
        \caption{#2}
        \label{fig:#3}
    \end{figure}
}

%% definitions
\newtheorem{theorem}{Theorem}
\newtheorem*{definition}{Definition}
\begin{document}
\customheading{\today}{Direct Proofs: P5b Subset System}{Fall 2024}{CS 30 - Discrete Maths}

\section*{}
\begin{parts}
    \part For any $j \geq 1$ and any $k \in \N$, argue that
        \begin{enumerate}
            \item if $F(j-1, k) = 1$ then $F(j, k) = 1$
            \begin{customsolutionbox}
                This implies that if there exists a subset $S' \subseteq \{1, 2, 3, \ldots j-1\}$ such that $\sum_{i \in S'} a_{i} = k$, then there exists a subset of $S \subseteq \{1, 2, 3, \ldots j\}$ such that $F(j, k) = 1$ \\

                To prove this, let's consider the following:
                \begin{gather}
                    \text{let} \hs S \subseteq \{1, 2, \ldots, j\} \\
                    \text{and let } S' \subseteq \{1, 2, \ldots, j-1\}
                \end{gather}
                \begin{itemize}
                    \item Case 1: if $j \not \in S$, then we can consider a subset $S' \subseteq S$ such that $\sum_{i \in S'} a_{i} = k$. This aligns with our assumption that $F(j-1, k) =1$
                    \item Case 2: if $j \in S$ then we consider the subsets of $S$. In this case, \textit{there exists} a subset $S' \subseteq S$ such that $\sum_{i \in S'} a_{i} = k$. Thus, it follows that $F(j, k) = 1$.
                \end{itemize}
            \end{customsolutionbox}

            \item if $F(j-1, k-a_{j}) = 1$ then $F(j, k) = 1$
            \begin{customsolutionbox}
                This implies that if there exists a subset $S' \subseteq S$, where $S = \{1, 2, \ldots, j\}$, such that $\sum_{i \in S'} a_{i} = k-a_{j}$, then it follows that $F(j, k) = 1$ \\

                To prove this consider the following:
                \begin{gather}
                    \text{let} \hs S \subseteq \{1, 2, \ldots, j\} \\
                \end{gather}
                \begin{enumerate}
                    \item Case 1: Suppose $j \not \in S$. If there exists a subset $S' \subseteq S$ such that $\sum_{i \in S'} a_{i} = k - a_{j}$, then it follows that $F(j-1, k-a_{j}) = 1$. Here the assumption is that $k$ is the sum of first $j$ elements.
                    \item Case 2: Suppose $j \in S$. If there exists a subset $S' \subseteq S$ such that $\sum_{i \in S'} a_{i}= k-a_{j}$, then adding $a_{j}$ gives $\sum_{i \in S'} a_{i} +a_{j} =k$. Therefore, it follows that $F(j, k) = 1$.
                \end{enumerate}
            \end{customsolutionbox}
        \end{enumerate}

    \part For any $j \geq 1$ and any $k \in \N$ argue that $F(j, k) = 1$ then either $F(j-1, k) = 1$ or $F(j-1, k-a_{j}) = 1$
        \begin{customsolutionbox}
            $F(j, k) =1$, implies that there exists subset of $S$, where $S \subseteq \{1, 2, \ldots, j\}$, such that $\sum_{i \in S'} a_{i} = k$. here $S' \subseteq S$ \\

            To prove if $F(j-1, k) = 1$ or $F(j-1, k-a_{j}) = 1$, consider the following:
            \begin{gather}
                \text{let} \hs S \subseteq \{1, 2, \ldots, j\} \\
            \end{gather}
            \begin{enumerate}
                \item If $F(j, k)= 1$ and $j \not \in S$, then $S \subseteq \{1, 2, \ldots j-1\}$. If there exists $S' \subseteq S$ such that $\sum_{i \in S'} a_{i} = k$ then this implies that $F(j-1, k) =1$.

                \item If $F(j, k)= 1$ and $j \in S$, then $S \subseteq \{1, 2, \ldots j\}$. If there exists $S' \subseteq S$ such that $\sum_{i \in S'} a_{i}= k- a_{j}$ then this implies $F(j-1, k-a_{j}) =1$.
            \end{enumerate}

            Conclusion: Both cases are valid because there exists a subset $S$ that satisfies the conditions.
        \end{customsolutionbox}
\end{parts}
\end{document}