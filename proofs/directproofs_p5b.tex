\documentclass[11pt]{exam}

%% import packages to avoid too many imports on main file
\usepackage{amsmath}
\usepackage{amssymb}
\usepackage{amsthm}
\usepackage{enumerate}
\usepackage{enumitem}
\usepackage{float}
\usepackage{graphicx}
\usepackage{hyperref}
\usepackage{ifthen}
\usepackage{mdframed}
\usepackage{tcolorbox}
\usepackage{xcolor}
\usepackage[sfdefault]{plex-sans}
\usepackage{tikz}


\newcommand{\R}{\ensuremath{\mathbb{R}}}
\newcommand{\N}{\ensuremath{\mathbb{N}\hspace{1.0mm}}}
\newcommand{\Z}{\ensuremath{\mathbb{Z}}}
\newcommand{\Q}{\ensuremath{\mathbb{Q}}}
\newcommand{\C}{\ensuremath{\mathbb{C}}}

% formatting
\newcommand{\hs}{\hspace{1.0mm}} % horizontal space; mostly in equations
\makeatletter
\def\exam@MakeFramed#1{\begin{mdframed}}
\def\endexam@MakeFramed{\end{mdframed}}
\makeatother


\newtcolorbox{customsolutionbox}[1][]{
    colback=white,
    colframe=gray!50!black,
    sharp corners=southwest,
    rounded corners=southeast,
    title=Solution,
    #1 % answer
}

% \newcommand{\customequation}[1]{
%     \vspace{5.0mm}
%     \begin{center}
%         \begin{align*}
%             #1
%         \end{align*}
%     \end{center}
%     % \vspace{5.0mm}
% }

%% credit statement
\newcommand{\creditstatement}[1][alone]{
    % \ifthenelse{\equal{#1}{alone}}{
    \begin{center}
        {\fontsize{14}{16}\selectfont \textbf{\textit{Credit statement: I worked alone on this assignment with reference to: #1} \\}}
    \end{center}
    % }{
    % \begin{center}
    %     {\fontsize{14}{16}\selectfont\textbf{\textit{Credit statement: I discussed the problems with #1, but worked alone on my solutions} \\}}
    % \end{center}
    % }
}

%% heading
\newcommand{\customheading}[4]{ % #1=Date, #2=Title
    \fbox{%
        \begin{minipage}{\textwidth}
        \begin{tikzpicture}[overlay, remember picture]
            \node[anchor=north west, font=\bfseries\itshape] at (0,0) {Abdibaset Bare};
            \node[anchor=north east, font=\bfseries] at (\textwidth,0) {#1};
            \node[anchor=north, font=\Large\bfseries\itshape] at (\textwidth/2,-1.1) {#2};
            \node[anchor=south west, font=\itshape] at (0,-3) {#3};
            \node[anchor=south east, font=\itshape] at (\textwidth,-3) {#4};
        \end{tikzpicture}
        \vspace{3cm}
        \end{minipage}%
    }
}

\newcommand{\customfigure}[3]{
    \begin{figure}[H]
        \centering
        \includegraphics[width=25mm]{#1}
        \caption{#2}
        \label{fig:#3}
    \end{figure}
}

%% definitions
\newtheorem{theorem}{Theorem}
\newtheorem*{definition}{Definition}
\begin{document}
\customheading{\today}{Direct Proofs: P5b Subset System}{Fall 2024}{CS 30 - Discrete Maths}

\section*{}
\begin{parts}
    \part For any $j \geq 1$ and any $k \in \N$, argue that
        \begin{enumerate}
            \item if $F(j-1, k) = 1$ then $F(j, k) = 1$
            \begin{customsolutionbox}

                To prove this, let's consider the following:
                \begin{gather}
                    \text{let's fix } S \subseteq \{1, 2, 3, \ldots j\} \\
                \end{gather}
                Suppose $j \not \in S$, we have a subset of $S^{*} = S \setminus \{j\} = \{1, 2, \ldots, j-1\}$,such that $\sum_{i \in S^{*}} a_{i} = k$. Therefore, since $S^{*} \subseteq S$, we conclude that $\sum_{i \in S} a_{i} = k$ hence $F(j, k) = 1$
            \end{customsolutionbox}

            \item if $F(j-1, k-a_{j}) = 1$ then $F(j, k) = 1$
            \begin{customsolutionbox}
                To prove this, let's consider the following:
                \begin{gather}
                    \text{let's fix } S \subseteq \{1, 2, 3, \ldots j\} \\
                \end{gather}
                Suppose $j \in S$ such that $\sum_{i \in S} a_{i} = k$ and $F(j, k) = 1$, there exists a subset $S^{*} = S \setminus \{j\} = \{1, 2, \ldots, j-1\}$ such that $\sum_{i \in S^{*}} a_{i} = k - a_j$ hence the assumption $F(j-1, k-a_j) = 1$ is true, and overall the implication is true.
            \end{customsolutionbox}
        \end{enumerate}

    \part For any $j \geq 1$ and any $k \in \N$ argue that $F(j, k) = 1$ then either $F(j-1, k) = 1$ or $F(j-1, k-a_{j}) = 1$
        \begin{customsolutionbox}

            To prove that $F(j-1, k) = 1$ or $F(j-1, k-a_{j}) = 1$, consider the following:
            \begin{gather}
                \text{let's fix } S \subseteq \{1, 2, 3, \ldots j\} \\
            \end{gather}
            \begin{enumerate}
                \item Case 1: $j \not \in S$ and $F(j, k) = 1$. In this case, the assumption is that there exists a subset $S^{*} = S \setminus \{j\} = \{1, 2, \ldots, j-1\}$ such that $\sum_{i \in S^{*}} a_{i} = k$. Therefore, we can conclude that $F(j-1, k) = 1$.
                \item Case 2. $j \in S$. In this case, we assume there exists a subset of $S$ such that $\sum_{i \in S} a_{i} = k$ hence $F(j, k) = 1$. We can conclude there is a subset $S^{*} = S \setminus \{j\} = \{1, 2, \ldots, j-1\}$ such that $\sum_{i \in S^{*}} a_i = k - a_j$ hence  $F(j-1, k-a_j) = 1$
            \end{enumerate}
            Thus, we have shown that if $F(j-1, k) = 1$ or $F(j-1, k-a_j) = 1$ then it follows that $F(j,k) = 1$
        \end{customsolutionbox}
\end{parts}
\end{document}