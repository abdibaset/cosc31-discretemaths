\documentclass[10pt]{exam}

%% import packages to avoid too many imports on main file
\usepackage{amsmath}
\usepackage{amssymb}
\usepackage{amsthm}
\usepackage{enumerate}
\usepackage{enumitem}
\usepackage{float}
\usepackage{graphicx}
\usepackage{hyperref}
\usepackage{ifthen}
\usepackage{lipsum}
\usepackage{mdframed}
\usepackage{tcolorbox}
\usepackage{xcolor}
\usepackage[sfdefault]{plex-sans}
\usepackage{tikz}

\tcbuselibrary{breakable}

\newcommand{\R}{\ensuremath{\mathbb{R}}}
\newcommand{\N}{\ensuremath{\mathbb{N}\hspace{1.0mm}}}
\newcommand{\Z}{\ensuremath{\mathbb{Z}}}
\newcommand{\Q}{\ensuremath{\mathbb{Q}}}
\newcommand{\C}{\ensuremath{\mathbb{C}}}

% formatting
\newcommand{\hs}{\hspace{1.0mm}} % horizontal space; mostly in equations
\makeatletter
\def\exam@MakeFramed#1{\begin{mdframed}}
\def\endexam@MakeFramed{\end{mdframed}}
\makeatother


\newtcolorbox{customsolutionbox}[1][]{
    colback=white,
    colframe=gray!50!black,
    sharp corners=southwest,
    rounded corners=southeast,
    title=Solution,
    breakable,
    #1 % answer
}

% \newcommand{\customequation}[1]{
%     \vspace{5.0mm}
%     \begin{center}
%         \begin{align*}
%             #1
%         \end{align*}
%     \end{center}
%     % \vspace{5.0mm}
% }

%% credit statement
\newcommand{\creditstatement}[1][alone]{
    % \ifthenelse{\equal{#1}{alone}}{
    \begin{center}
        {\fontsize{14}{16}\selectfont \textbf{\textit{Credit statement: I worked alone on this assignment with reference to: #1} \\}}
    \end{center}
    % }{
    % \begin{center}
    %     {\fontsize{14}{16}\selectfont\textbf{\textit{Credit statement: I discussed the problems with #1, but worked alone on my solutions} \\}}
    % \end{center}
    % }
}

%% heading
\newcommand{\customheading}[4]{ % #1=Date, #2=Title
    \fbox{%
        \begin{minipage}{\textwidth}
        \begin{tikzpicture}[overlay, remember picture]
            \node[anchor=north west, font=\bfseries\itshape] at (0,0) {Abdibaset Bare};
            \node[anchor=north east, font=\bfseries] at (\textwidth,0) {#1};
            \node[anchor=north, font=\Large\bfseries\itshape] at (\textwidth/2,-1.1) {#2};
            \node[anchor=south west, font=\itshape] at (0,-3) {#3};
            \node[anchor=south east, font=\itshape] at (\textwidth,-3) {#4};
        \end{tikzpicture}
        \vspace{3cm}
        \end{minipage}%
    }
}

\newcommand{\customfigure}[3]{
    \begin{figure}[H]
        \centering
        \includegraphics[width=25mm]{#1}
        \caption{#2}
        \label{fig:#3}
    \end{figure}
}

%% definitions
\newtheorem{theorem}{Theorem}
\newtheorem*{definition}{Definition}
\begin{document}
\customheading{\today}{Problem 12: Proof by PHP}{Fall 2024}{CS30 - Discrete Maths}
\section*{}
\begin{parts}
    \part Prove that any odd number $n$ which isn't divisible by 5 has a multiple which, in decimal notation, looks like a string of consecutive 1s. Next, write code which takes $n = 1729$ is fed to your code. \\

    For example, $n=7, 111111 = 7 \cdot 15873$ is one such multiple, and your code should return the same.

    \textit{Hint: if $n$ is odd not divisible by 5(and so relatively prime with 10), and a number $xyz0$ is a multiple of $n$, then $xyz$ is also a multiple of $n$.}

    \begin{customsolutionbox}
        To generate a number of a string of 1s denoted as $m_{k}$, we can use the following:
        \begin{gather}
            m_{k} = \frac{10^{k}-1}{9} \\
            k = 1; \frac{10^{1}-1}{9} = 1\\
            k = 2; \frac{10^{2}-1}{9} = 11 \\
            k = 3; \frac{10^{3}-1}{9} = 111
        \end{gather}
        As shown, each $m_{k}$ represents a number made up of $k$ consecutive 1s. Since $n$ is an odd number not divisible by 5, it is relatively prime to 10, which means $m_k$ cannot be congruent to 0 modulo 5. \\

        Suppose $n$ is divisible by $m_{k}$ then:
        \begin{gather}
            m_{k}\mod n = 0 \\
            \frac{10^{k}-1}{9} \mod n = 0
        \end{gather}

        To prove that a $k$ exists such that remainders start repeating when $m_{k}$ divided by $n$ we can apply pigeon-hole prinicple. The set of possible remainders is $\{0, 1, 2, \ldots, n-1\}$ are the pigeon-holes in this case, and $m_{k}$, i.e., the string of 1s is the number of holes.By pigeonhole prinicple, since there are only $n$ possible remainders, and we can create an inifinite sequences of remainders as $k$ increases, at least two of these remainders must repeat.

        Suppose the remainders repeat at $k^{*}$ and $k$ when $k^{*} < k$ then:
        \begin{gather}
            m_{k^{*}} \mod n \equiv m_{k} \mod n; \\
        \end{gather}
        which implies
        \begin{gather}
            m_{k} - m_{k^{*}} \mod n = 0
        \end{gather}
        Thus, we conclude that there exists a multiple of $n$ which is a string of consecutive 1s for a given $k$
    \end{customsolutionbox}

    \subsection{code}
    \begin{verbatim}
        def PigeoholePrinciple(oddNumber: int):
            stringOf1s = 1 ## (10^1 -1)/9 where k = 1
            k = 2

            while stringOf1s % oddNumber != 0:
                stringOf1s = ((10**k) - 1) // 9
                k += 1
            return int(stringOf1s)

        print(PigeoholePrinciple(7), `has length ', len(str(PigeoholePrinciple(7))))
        print(PigeoholePrinciple(9), `has length ', len(str(PigeoholePrinciple(9))))
        print(PigeoholePrinciple(1729), `has length ', len(str(PigeoholePrinciple(1729))))
    \end{verbatim}

    Therefore when $n = 1729$, a string 1s of length 18 will be divisible by $n$
\end{parts}

\end{document}