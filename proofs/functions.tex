\documentclass[11pt]{article}
\usepackage{graphicx}
%% import packages to avoid too many imports on main file
\usepackage{amsmath}
\usepackage{amssymb}
\usepackage{amsthm}
\usepackage{enumerate}
\usepackage{enumitem}
\usepackage{float}
\usepackage{graphicx}
\usepackage{hyperref}
\usepackage{ifthen}
\usepackage{mdframed}
\usepackage{tcolorbox}
\usepackage{xcolor}
\usepackage[sfdefault]{plex-sans}
\usepackage{tikz}


\newcommand{\R}{\ensuremath{\mathbb{R}}}
\newcommand{\N}{\ensuremath{\mathbb{N}\hspace{1.0mm}}}
\newcommand{\Z}{\ensuremath{\mathbb{Z}}}
\newcommand{\Q}{\ensuremath{\mathbb{Q}}}
\newcommand{\C}{\ensuremath{\mathbb{C}}}

% formatting
\newcommand{\hs}{\hspace{1.0mm}} % horizontal space; mostly in equations
\makeatletter
\def\exam@MakeFramed#1{\begin{mdframed}}
\def\endexam@MakeFramed{\end{mdframed}}
\makeatother


\newtcolorbox{customsolutionbox}[1][]{
    colback=white,
    colframe=gray!50!black,
    sharp corners=southwest,
    rounded corners=southeast,
    title=Solution,
    #1 % answer
}

% \newcommand{\customequation}[1]{
%     \vspace{5.0mm}
%     \begin{center}
%         \begin{align*}
%             #1
%         \end{align*}
%     \end{center}
%     % \vspace{5.0mm}
% }

%% credit statement
\newcommand{\creditstatement}[1][alone]{
    % \ifthenelse{\equal{#1}{alone}}{
    \begin{center}
        {\fontsize{14}{16}\selectfont \textbf{\textit{Credit statement: I worked alone on this assignment with reference to: #1} \\}}
    \end{center}
    % }{
    % \begin{center}
    %     {\fontsize{14}{16}\selectfont\textbf{\textit{Credit statement: I discussed the problems with #1, but worked alone on my solutions} \\}}
    % \end{center}
    % }
}

%% heading
\newcommand{\customheading}[4]{ % #1=Date, #2=Title
    \fbox{%
        \begin{minipage}{\textwidth}
        \begin{tikzpicture}[overlay, remember picture]
            \node[anchor=north west, font=\bfseries\itshape] at (0,0) {Abdibaset Bare};
            \node[anchor=north east, font=\bfseries] at (\textwidth,0) {#1};
            \node[anchor=north, font=\Large\bfseries\itshape] at (\textwidth/2,-1.1) {#2};
            \node[anchor=south west, font=\itshape] at (0,-3) {#3};
            \node[anchor=south east, font=\itshape] at (\textwidth,-3) {#4};
        \end{tikzpicture}
        \vspace{3cm}
        \end{minipage}%
    }
}

\newcommand{\customfigure}[3]{
    \begin{figure}[H]
        \centering
        \includegraphics[width=25mm]{#1}
        \caption{#2}
        \label{fig:#3}
    \end{figure}
}

%% definitions
\newtheorem{theorem}{Theorem}
\newtheorem*{definition}{Definition}
\begin{document}
\customheading{\today}{Proofs: Function types}{Fall 2024}{CS30 - Discrete Maths}

\section*{Title}
\begin{enumerate}
    \item \textbf{bijection}:[1:1 correspondence/invertible] $\rightarrow$ \textit{Function} such that each element in the second set(codomain - set of all possible output values) is mapped to \textit{exactly} one element of the element. \textit{Note: a function is bijective if it is both injective and surjective. Therefore, to prove a function is a bijection you need to prove that it is a Surjection and an injection}
    \customfigure{../images/bijective.png}{bijective function: 1:1 mapping}{bijection} Example:
    \begin{equation}
        f(x) = 5x+8
    \end{equation}
    \begin{equation}
        \forall y \in Y, \exists !x \in X: y = f(x)
    \end{equation}
    $\exists !x$z - means there exists exactly one

    \item \textbf{Injection}: \textit{function} that maps each element of the codomain to by \textit{at most} one element of the domain
    \customfigure{../images/injective.png}{injective function}{injection}
    \begin{equation}
        \forall x, x' \in X: f(x) = f(x') \Rightarrow x = x'
    \end{equation}
    This means for any elements $x$ and $x'$, have the same output under a function it means $x=x'$. \textit{Equal inputs mean equal outputs}
    \textbf{Proof technique}: Start by assuming that $f(x)=f(x')$, then prove that $x=x'$.
    \begin{equation}
        \begin{aligned}
            f(x) = 5x + 8 \\
            \textnormal{pick any two unequal real number} \hs x_{1} \not = x_{2} \hs s.t. \\
            5x_{1} + 8 \not = 5x_{2} + 8
        \end{aligned}
    \end{equation}
    \textbf{another proof technique}: Proving that two elements in the domain maps to the same element in the codomain that's $5x_{1}^2 + 8 = 5x_{2}^2+8; x_{1} \not = x_{2}$ \\
    \begin{equation}
        \begin{aligned}
            f(x) = 5x^2 + 8 \geq 5x^2 \geq 0 \\
            f(-1) = f(1); \textnormal{not injective}
        \end{aligned}
    \end{equation}

    \textbf{Composition of two injective} is injective. i.e if $f: A \rightarrow B$ and $g:B \rightarrow A$ are both injective then $(g \cdot f)(x)$ is injective. \\ To show this, we need to show that for $a_{1} \not = a_{2} \in A, \hs s.t. \hs (g \cdot f)(a_{1}) \not = (g \cdot f)(a_{2})$. \\ Consider any $a_{1}, a_{2}\in A$ such that $a_{1} \not = a_{2}$
    Let $b_{1} = f(a_{1}) \hs and \hs b_{2} = f(a_{2})$. Since $a_{1} \not = a_{2} \hs and \hs f$ is injective, $b_{1} \not = b_{2}$. Since $g$ is injective, $b_{1} \not = b_{2}$ implies $g(b_{1}) \not = g(b_{2})$ that's  $(g \cdot f)(a_{1}) \not = (g \cdot f)(a_{2})$

    \item \textbf{Surjection}: \textit{function} that maps each element in the codomain to at least one element in the domain.
    \customfigure{../images/surjective.png}{surjective function}{surjective}
    \begin{equation}
        \forall y \in Y, \exists x \in X: y = f(x)
    \end{equation}
    In plain english: $f: X \rightarrow Y$ is surjective if for every element in the codomain, there exists in the domain, s.t. $f(x) = y$ \\
    \textbf{Proof}: take any $y \in \R$, show that we can take an x value s.t. $f(x) =y$
    \begin{equation}
        \begin{aligned}
            f(x) = 5x+8 \\
            y = 5x + 8 \\
            x = \frac{y-8}{5} \\
            f(x) = 5 \cdot{\frac{y-8}{5} + 8} \\
            f(x) = y - 8 + 8 \\
            f(x) = y
        \end{aligned}
    \end{equation}

    \textbf{Another proof style}: Proving that not all the elements in the codomain are mapped to the doamin

    \begin{equation}
        f(x) = 5x^2 + 8 \geq 5x^2 \geq 0
    \end{equation}
    \textbf{Composition} Prove that if $f:A \rightarrow B$ and $g:B \rightarrow C$ are surjective, then $(g \cdot f)(x)$ is surjective

    \textit{Goal: Prove for any $c \in C$ there exists $a \in A$ such that
    $(g \cdot f)(a) = c$} \\ \\

    Fix $c \in C$. Since $g: B \rightarrow C$ is surjective, given $c \in C$, there exists $b \in B$ such that $g(b) = c$. Since, $f: A \rightarrow B$, given this $b \in B$, there exists $a \in A$ such that $f(a) = b$. That is, $g(f(a)) = g(b) = c$. Thus, $(g \cdot f)(a) = c$. This shows $(g \cdot f)(x)$ is surjective.
    \item \textbf{general function}
\end{enumerate}
\end{document}