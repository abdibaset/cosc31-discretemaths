\documentclass[11pt]{exam}

%% import packages to avoid too many imports on main file
\usepackage{amsmath}
\usepackage{amssymb}
\usepackage{amsthm}
\usepackage{enumerate}
\usepackage{enumitem}
\usepackage{float}
\usepackage{graphicx}
\usepackage{hyperref}
\usepackage{ifthen}
\usepackage{mdframed}
\usepackage{tcolorbox}
\usepackage{xcolor}
\usepackage[sfdefault]{plex-sans}
\usepackage{tikz}


\newcommand{\R}{\ensuremath{\mathbb{R}}}
\newcommand{\N}{\ensuremath{\mathbb{N}\hspace{1.0mm}}}
\newcommand{\Z}{\ensuremath{\mathbb{Z}}}
\newcommand{\Q}{\ensuremath{\mathbb{Q}}}
\newcommand{\C}{\ensuremath{\mathbb{C}}}

% formatting
\newcommand{\hs}{\hspace{1.0mm}} % horizontal space; mostly in equations
\makeatletter
\def\exam@MakeFramed#1{\begin{mdframed}}
\def\endexam@MakeFramed{\end{mdframed}}
\makeatother


\newtcolorbox{customsolutionbox}[1][]{
    colback=white,
    colframe=gray!50!black,
    sharp corners=southwest,
    rounded corners=southeast,
    title=Solution,
    #1 % answer
}

% \newcommand{\customequation}[1]{
%     \vspace{5.0mm}
%     \begin{center}
%         \begin{align*}
%             #1
%         \end{align*}
%     \end{center}
%     % \vspace{5.0mm}
% }

%% credit statement
\newcommand{\creditstatement}[1][alone]{
    % \ifthenelse{\equal{#1}{alone}}{
    \begin{center}
        {\fontsize{14}{16}\selectfont \textbf{\textit{Credit statement: I worked alone on this assignment with reference to: #1} \\}}
    \end{center}
    % }{
    % \begin{center}
    %     {\fontsize{14}{16}\selectfont\textbf{\textit{Credit statement: I discussed the problems with #1, but worked alone on my solutions} \\}}
    % \end{center}
    % }
}

%% heading
\newcommand{\customheading}[4]{ % #1=Date, #2=Title
    \fbox{%
        \begin{minipage}{\textwidth}
        \begin{tikzpicture}[overlay, remember picture]
            \node[anchor=north west, font=\bfseries\itshape] at (0,0) {Abdibaset Bare};
            \node[anchor=north east, font=\bfseries] at (\textwidth,0) {#1};
            \node[anchor=north, font=\Large\bfseries\itshape] at (\textwidth/2,-1.1) {#2};
            \node[anchor=south west, font=\itshape] at (0,-3) {#3};
            \node[anchor=south east, font=\itshape] at (\textwidth,-3) {#4};
        \end{tikzpicture}
        \vspace{3cm}
        \end{minipage}%
    }
}

\newcommand{\customfigure}[3]{
    \begin{figure}[H]
        \centering
        \includegraphics[width=25mm]{#1}
        \caption{#2}
        \label{fig:#3}
    \end{figure}
}

%% definitions
\newtheorem{theorem}{Theorem}
\newtheorem*{definition}{Definition}
\begin{document}
\customheading{\today}{Problem 19: Trinomial Expansion}{Fall 2024}{CS 30 - Discrete Maths}

\section*{}
After understanding the binomial expansion done in class, you are ready to figure out the \textit{trinomial
expansion}. Consider the expansion of the expression $(x+y+z)^n$ as a polynomial on three variables
with various monomials. For each of these give a line of reasoning.
\begin{parts}
    \part Write down the set of monomials you see in set-builder notation?
        \begin{customsolutionbox}
            Let $A$ be the set of monomials. Since we have $3$ variables, the monomial will be of the form $x^ay^bz^c$, where $0 \leq a, b, c \leq n$, and $a+b+c = n$. $A's$ set builder notation is as follows:
            \begin{align*}
                A &= \{x^ay^bz^c: 0 \leq a, b, c \leq n, \hs  a+b+c = n\}
            \end{align*}
        \end{customsolutionbox}

    \part What is the cardinality of the above set of monomials?
        \begin{customsolutionbox}
            To find the cardinality of monomials, we can use the stars and bars method. The number of non-negative integer solutions to the equation $a + b + c = n \equiv k$ and $n = 3 \text{ i.e. } (a, b, c)$ is given by:
            \begin{align*}
                {n + k - 1 \choose k} &= {3 + n - 1 \choose n} \\
                &= \frac{(n+2)!}{n!\cdot(3-1)!}  \\
                &= \frac{(n+2)!}{n!\cdot 2!} \\
                &= \frac{(n+2) \cdot (n+1) \cdot n!}{n!\cdot 2!} \\
                &= \frac{(n+2)\cdot(n+1)}{2}
            \end{align*}
        \end{customsolutionbox}
    \part What is the coefficient, in the trinomial expansion, of every monomial?
        \begin{customsolutionbox}
            The coefficient of each monomial can be represented by $\binom{n}{k}$ where $k = a,b,c$ and $n = a+b+c$:
            \begin{align}
                \text{Num coefficients } &=\binom{n}{a,b,c} \\
                &= \frac{n!}{a!b!c! \cdot (n - (a+b+c))} \\
                &= \frac{n!}{a!b!c! \cdot 1} \\
                &= \frac{n!}{a!b!c!}
            \end{align}
            We do so, because the order in which $xyz$ appears doesn't matter.
        \end{customsolutionbox}
\end{parts}
\end{document}