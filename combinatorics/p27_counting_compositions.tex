\documentclass[11pt]{exam}

%% import packages to avoid too many imports on main file
\usepackage{amsmath}
\usepackage{amssymb}
\usepackage{amsthm}
\usepackage{enumerate}
\usepackage{enumitem}
\usepackage{float}
\usepackage{graphicx}
\usepackage{hyperref}
\usepackage{ifthen}
\usepackage{mdframed}
\usepackage{tcolorbox}
\usepackage{xcolor}
\usepackage[sfdefault]{plex-sans}
\usepackage{tikz}


\newcommand{\R}{\ensuremath{\mathbb{R}}}
\newcommand{\N}{\ensuremath{\mathbb{N}\hspace{1.0mm}}}
\newcommand{\Z}{\ensuremath{\mathbb{Z}}}
\newcommand{\Q}{\ensuremath{\mathbb{Q}}}
\newcommand{\C}{\ensuremath{\mathbb{C}}}

% formatting
\newcommand{\hs}{\hspace{1.0mm}} % horizontal space; mostly in equations
\makeatletter
\def\exam@MakeFramed#1{\begin{mdframed}}
\def\endexam@MakeFramed{\end{mdframed}}
\makeatother


\newtcolorbox{customsolutionbox}[1][]{
    colback=white,
    colframe=gray!50!black,
    sharp corners=southwest,
    rounded corners=southeast,
    title=Solution,
    #1 % answer
}

% \newcommand{\customequation}[1]{
%     \vspace{5.0mm}
%     \begin{center}
%         \begin{align*}
%             #1
%         \end{align*}
%     \end{center}
%     % \vspace{5.0mm}
% }

%% credit statement
\newcommand{\creditstatement}[1][alone]{
    % \ifthenelse{\equal{#1}{alone}}{
    \begin{center}
        {\fontsize{14}{16}\selectfont \textbf{\textit{Credit statement: I worked alone on this assignment with reference to: #1} \\}}
    \end{center}
    % }{
    % \begin{center}
    %     {\fontsize{14}{16}\selectfont\textbf{\textit{Credit statement: I discussed the problems with #1, but worked alone on my solutions} \\}}
    % \end{center}
    % }
}

%% heading
\newcommand{\customheading}[4]{ % #1=Date, #2=Title
    \fbox{%
        \begin{minipage}{\textwidth}
        \begin{tikzpicture}[overlay, remember picture]
            \node[anchor=north west, font=\bfseries\itshape] at (0,0) {Abdibaset Bare};
            \node[anchor=north east, font=\bfseries] at (\textwidth,0) {#1};
            \node[anchor=north, font=\Large\bfseries\itshape] at (\textwidth/2,-1.1) {#2};
            \node[anchor=south west, font=\itshape] at (0,-3) {#3};
            \node[anchor=south east, font=\itshape] at (\textwidth,-3) {#4};
        \end{tikzpicture}
        \vspace{3cm}
        \end{minipage}%
    }
}

\newcommand{\customfigure}[3]{
    \begin{figure}[H]
        \centering
        \includegraphics[width=25mm]{#1}
        \caption{#2}
        \label{fig:#3}
    \end{figure}
}

%% definitions
\newtheorem{theorem}{Theorem}
\newtheorem*{definition}{Definition}
\begin{document}
\customheading{\today}{Problem 27:Counting Composition}{Fall 2024}{CS 30 - Discrete Maths}
\section*{}
\begin{parts}
    \part Given a number $n$, a composition is a sequence $(a_1, a_2, \ldots)$ whose entries are positive integers which
    sum up to $n$. For example, for $n = 5$, the sequences $(5), (1, 4), (4, 1)$, and $(2, 3)$ are four different
    sequences of the number $5$. How many compositions does the number $n$ have? Your answer should be a simple function of $n$.
    \begin{customsolutionbox}
        We can get the number of possible sequences using the stars and bars method. Suppose we have $n$ stars, and $n-1$ possible bars(corresponding to number of gaps between stars) which partitions the stars up to $n-1$ groups. Each star represents 1 unit. The sum of the stars bounded by two dividers forms an element of the sequence. For example:
        \begin{align}
            0 \text{ bars } \Rightarrow ***** &= (5) \sum{(5)} = 5 \\
            2  \text{ bars } \Rightarrow **|**|* &= (2, 2, 1) \sum{(2, 2, 1)} = 5 \\
            3 \text{ bars } \Rightarrow *|*|*|** &= (1, 1, 1, 2) \sum{(1, 1, 1, 2)} = 5 \\
            4 \text{ bars } \Rightarrow *|*|*|*|* &= (1, 1,1, 1, 1) \sum{(1, 1, 1, 1, 1)} = 5
        \end{align}
        Consider the bars as a string of length $n-1$. If the $i^{th}$ bar where $1 \leq i \leq n -1$ is included in partitioning the stars into groups, then it is represented as $1$, otherwise $0$. For above cases we have:
        \begin{align}
            0  \text{ bars } \Rightarrow ***** &= 0000 \\
            2 \text{ bars } \Rightarrow **|**|* &= 0101 \\
            3  \text{ bars } \Rightarrow *|*|*|** &= 1110 \\
            4  \text{ bars } \Rightarrow *|*|*|*|* &= 1111
        \end{align}
        Therefore, as illustrated above we can establish a bijective function that maps the set of bars (created by the bars) to a set of binary strings of length $n-1$. Each bit has $S=\{0, 1\}$ possibilities. The total number of sequences can be given by the size of the set $|\{0, 1\}^{n-1}|$ which is $2^{n-1}$.
    \end{customsolutionbox}

\end{parts}
\end{document}