\documentclass[9pt]{exam}

%% import packages to avoid too many imports on main file
\usepackage{amsmath}
\usepackage{amssymb}
\usepackage{amsthm}
\usepackage{enumerate}
\usepackage{enumitem}
\usepackage{float}
\usepackage{graphicx}
\usepackage{hyperref}
\usepackage{ifthen}
\usepackage{mdframed}
\usepackage{tcolorbox}
\usepackage{xcolor}
\usepackage[sfdefault]{plex-sans}
\usepackage{tikz}


\newcommand{\R}{\ensuremath{\mathbb{R}}}
\newcommand{\N}{\ensuremath{\mathbb{N}\hspace{1.0mm}}}
\newcommand{\Z}{\ensuremath{\mathbb{Z}}}
\newcommand{\Q}{\ensuremath{\mathbb{Q}}}
\newcommand{\C}{\ensuremath{\mathbb{C}}}

% formatting
\newcommand{\hs}{\hspace{1.0mm}} % horizontal space; mostly in equations
\makeatletter
\def\exam@MakeFramed#1{\begin{mdframed}}
\def\endexam@MakeFramed{\end{mdframed}}
\makeatother


\newtcolorbox{customsolutionbox}[1][]{
    colback=white,
    colframe=gray!50!black,
    sharp corners=southwest,
    rounded corners=southeast,
    title=Solution,
    #1 % answer
}

% \newcommand{\customequation}[1]{
%     \vspace{5.0mm}
%     \begin{center}
%         \begin{align*}
%             #1
%         \end{align*}
%     \end{center}
%     % \vspace{5.0mm}
% }

%% credit statement
\newcommand{\creditstatement}[1][alone]{
    % \ifthenelse{\equal{#1}{alone}}{
    \begin{center}
        {\fontsize{14}{16}\selectfont \textbf{\textit{Credit statement: I worked alone on this assignment with reference to: #1} \\}}
    \end{center}
    % }{
    % \begin{center}
    %     {\fontsize{14}{16}\selectfont\textbf{\textit{Credit statement: I discussed the problems with #1, but worked alone on my solutions} \\}}
    % \end{center}
    % }
}

%% heading
\newcommand{\customheading}[4]{ % #1=Date, #2=Title
    \fbox{%
        \begin{minipage}{\textwidth}
        \begin{tikzpicture}[overlay, remember picture]
            \node[anchor=north west, font=\bfseries\itshape] at (0,0) {Abdibaset Bare};
            \node[anchor=north east, font=\bfseries] at (\textwidth,0) {#1};
            \node[anchor=north, font=\Large\bfseries\itshape] at (\textwidth/2,-1.1) {#2};
            \node[anchor=south west, font=\itshape] at (0,-3) {#3};
            \node[anchor=south east, font=\itshape] at (\textwidth,-3) {#4};
        \end{tikzpicture}
        \vspace{3cm}
        \end{minipage}%
    }
}

\newcommand{\customfigure}[3]{
    \begin{figure}[H]
        \centering
        \includegraphics[width=25mm]{#1}
        \caption{#2}
        \label{fig:#3}
    \end{figure}
}

%% definitions
\newtheorem{theorem}{Theorem}
\newtheorem*{definition}{Definition}
\begin{document}
\customheading{\today}{Problem 3: Number of functions}{Fall 2024}{CS30 - Discrete Maths}

\section*{}
Give short and precise reasons for your answers below. All functions mentioned below are of the form $f: A \rightarrow B$ for some finite sets $A$ and $B$.
\begin{parts}
    \part How many functions are where $|A| = n$ and $|B| = m$
        \begin{customsolutionbox}
            * If there are $n$ elements in the domain, i.e., set $|A| = n$, each element in the domain has $m$ options to map to in the codomain, i.e., $|B|=m$. Therefore, by the product priniciple, the number of functions are:
            \begin{gather}
                \text{Number of functions } = m^n
            \end{gather}
        \end{customsolutionbox}

    \part How many \textit{bijective} functions are there where $|A| = n$ and $|B| = n$
        \begin{customsolutionbox}
            * A bijective function has to be both injective(one-to-one) and surjective. We know that that an injective function is a surjective function, but surjective function may not always be injective. Therefore:
            \begin{itemize}
                \item $1^{st}$ element in the domain, $A$, has $n$ options to map to in the codomain, $B$.
                \item $2^{nd}$ element in the domain, $A$, has $n-1$ options to map to in the codomain.
                \item $3^{rd}$ element in the domain, $A$, has $n-2$ options to map to in the codomain.
            \end{itemize}
            Therefore, by the product priniciple, the number of \textit{bijective} functions are:
            \begin{gather}
                \text{Number of \textit{bijective} functions } = n! \equiv n \cdot (n-1) \cdot (n-2) \cdot \ldots \cdot 1
            \end{gather}
        \end{customsolutionbox}

    \part How many injective functions are there when $|A| =n $ and $|B|=m$
        \begin{customsolutionbox}
            * An injective function maps every element in the domain, $A$ to exactly one element in the codomain, $B$, and two element in the domain cannot map to the same elemnent in the codomain. Therefore, $n \leq m$. hence by the division priniciple, the number of injective functions are:
            \begin{itemize}
                \item $1^{st}$ element in the domain, $A$, has $m$ options in the codomain, $B$, to map to.
                \item $2^{nd}$ element in the domain, $A$, has $m-1$ options in the codomain, $B$, to map to.
                \item $3^{rd}$ element in the domain, $A$, has $m-2$ options in the codomain $B$ to map to.
                \item \ldots
                \item Last element in the domain, $A$ maps to $m-n+1$ element in the codomain, $B$, therefore, the number of injective functions are:
            \end{itemize}
            \begin{gather}
                \text{Number of \textit{injective} functions } = \frac{m!}{(m-n)!}
            \end{gather}
        \end{customsolutionbox}

    \part How many surjective functions are there when $|A| = n$ and $|B| = 2$
        \begin{customsolutionbox}
            * A surjective function has every element in the codomain has at least one element in the domain that maps to it. Therefore, for every element in the domain has $2$ options in the codomain to map to, hence:
            \begin{gather}
                \text{Number of functions } = 2^n
            \end{gather}
            However, there are two possibilities where every element in the domain maps to a single element in the codomain making the function non-surjective. Hence the total number of surjective functions are:
            \begin{gather}
                \text{Number of surjective functions } = 2^n - 2
            \end{gather}
        \end{customsolutionbox}
\end{parts}
\end{document}