\documentclass[9pt]{exam}

%% import packages to avoid too many imports on main file
\usepackage{amsmath}
\usepackage{amssymb}
\usepackage{amsthm}
\usepackage{enumerate}
\usepackage{enumitem}
\usepackage{float}
\usepackage{graphicx}
\usepackage{hyperref}
\usepackage{ifthen}
\usepackage{lipsum}
\usepackage{mdframed}
\usepackage{tcolorbox}
\usepackage{xcolor}
\usepackage[sfdefault]{plex-sans}
\usepackage{tikz}

\tcbuselibrary{breakable}

\newcommand{\R}{\ensuremath{\mathbb{R}}}
\newcommand{\N}{\ensuremath{\mathbb{N}\hspace{1.0mm}}}
\newcommand{\Z}{\ensuremath{\mathbb{Z}}}
\newcommand{\Q}{\ensuremath{\mathbb{Q}}}
\newcommand{\C}{\ensuremath{\mathbb{C}}}

% formatting
\newcommand{\hs}{\hspace{1.0mm}} % horizontal space; mostly in equations
\makeatletter
\def\exam@MakeFramed#1{\begin{mdframed}}
\def\endexam@MakeFramed{\end{mdframed}}
\makeatother


\newtcolorbox{customsolutionbox}[1][]{
    colback=white,
    colframe=gray!50!black,
    sharp corners=southwest,
    rounded corners=southeast,
    title=Solution,
    breakable,
    #1 % answer
}

% \newcommand{\customequation}[1]{
%     \vspace{5.0mm}
%     \begin{center}
%         \begin{align*}
%             #1
%         \end{align*}
%     \end{center}
%     % \vspace{5.0mm}
% }

%% credit statement
\newcommand{\creditstatement}[1][alone]{
    % \ifthenelse{\equal{#1}{alone}}{
    \begin{center}
        {\fontsize{14}{16}\selectfont \textbf{\textit{Credit statement: I worked alone on this assignment with reference to: #1} \\}}
    \end{center}
    % }{
    % \begin{center}
    %     {\fontsize{14}{16}\selectfont\textbf{\textit{Credit statement: I discussed the problems with #1, but worked alone on my solutions} \\}}
    % \end{center}
    % }
}

%% heading
\newcommand{\customheading}[4]{ % #1=Date, #2=Title
    \fbox{%
        \begin{minipage}{\textwidth}
        \begin{tikzpicture}[overlay, remember picture]
            \node[anchor=north west, font=\bfseries\itshape] at (0,0) {Abdibaset Bare};
            \node[anchor=north east, font=\bfseries] at (\textwidth,0) {#1};
            \node[anchor=north, font=\Large\bfseries\itshape] at (\textwidth/2,-1.1) {#2};
            \node[anchor=south west, font=\itshape] at (0,-3) {#3};
            \node[anchor=south east, font=\itshape] at (\textwidth,-3) {#4};
        \end{tikzpicture}
        \vspace{3cm}
        \end{minipage}%
    }
}

\newcommand{\customfigure}[3]{
    \begin{figure}[H]
        \centering
        \includegraphics[width=25mm]{#1}
        \caption{#2}
        \label{fig:#3}
    \end{figure}
}

%% definitions
\newtheorem{theorem}{Theorem}
\newtheorem*{definition}{Definition}
\begin{document}
\customheading{\today}{Problem 10: Subsequences and Substrings}{Fall 2024}{CS 30: Discrete Maths}
\section*{}
In both problems, obtain a solution via a bijection from/to a set whose cardinality you already know/can
easily figure out. Describe the bijection well and say a line or two why it is a bijection (i.e, injection &
surjection).

\begin{parts}
    \part Given a string $s[1 : n]$ with n characters, a \textit{substring} $t$ is a contiguous part of the string $s$. For example, if the string $s$ = banana, then ban, nana, the empty string $\bot$, the while string banana are all valid substrings. The strings bnn or apple are not. It is important for this question that the indices of s where the substring appears matter, and so, an appearing in locations 2 and 3 of
    $s$ is considered a different substring than the an appearing in locations 4 and 5 of s.
    How many substrings does a string $s[1 : n]$ with $n$ characters have? \\

    Your answer should be a function of $n$. For concreteness, write down the numerical answer for $n = 5$

    \begin{customsolutionbox}
       There is a bijective function that maps the set of substrings to their corresponding start and end indices in $s[1:n]$. Consider the set of substrings as the domain which is mapped to a pair of start and end indices $(i, j)$, where $1 \leq i \leq j \leq n$. \\

       \begin{enumerate}
        \item  If $(i = j)$, then the substring's length is 1
        \item  If $(i < j)$, then the substring's length is at least 2
       \end{enumerate}
       Since, we have $n$ elements corresponding to the length of $s[1:n]$, we can derive all possible substrings by considering the following cases:
       \begin{enumerate}
        \item $k = 2$ when $i < j$, that's we have two unique indices, start and end
        \item $k = 1$ when $i = j$
       \end{enumerate}
       Hence, the total number of substrings including the empty string is given:
        \begin{gather}
            \text{All possible substrings } = {n \choose 2} + {n \choose 1} + \bot \\
        \end{gather}
        When $n = 5$:
        \begin{gather}
            \frac{5!}{2!\cdot(5-2)!} +  \frac{5!}{1!\cdot(5-1)!}  + 1 =  16
        \end{gather}
    \end{customsolutionbox}

    \part Given a string $s[1 : n]$ with $n$ characters, a \textit{subsequence} t is a string whose characters appear in $s$ in the same order, however, not necessarily contiguously. For example, if the string $s$ = banana, then all substrings defined in part (a) are valid subsequences. But so is baa since the characters
    b, a, a do appear in banana in that order in positions $1$, $2$ and $4$. Note that positions $1$, $2$ and $6$ also give the \textit{subsequence} baa and for this question this is considered to be a different subsequence. The string baan is also a valid subsequence appearing in positions $1, 2, 4, 5$. But baann is not a \textit{subsequence}. \\

    How many subsequences does a string $s[1 : n]$ with n characters have? Your answer should be a
    function of $n$. For concreteness, write down the numerical answer for $n = 5$.
    \begin{customsolutionbox}
        We can consider each character in $s[1:n]$ having two possibilities $S = \{0, 1\}$, where it can either be included(1) or excluded(0) from $s[1:n]$ to form a subsequence. There exists bijective function whose domain is the set of subsequences, and the co-domain is a set of binary sequences. This bijective function maps each subsequence of $s[1:n]$ to a unique binary sequence of length $n$ in the co-domain, where each bit represents whether the corresponding character is included or not \\

        Therefore, the total number of possible subsequences is given by the size of the set $|S = \{0, 1\}^n|$, which is  $2^n$. This accounts for all possible subsequences, including the empty string. \\

        When $n = 5$
        \begin{gather}
            \text{All possible subsequences } = 2^5 = 32
        \end{gather}
    \end{customsolutionbox}
\end{parts}
\end{document}