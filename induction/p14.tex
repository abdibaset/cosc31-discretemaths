\documentclass[11pt]{exam}

%% import packages to avoid too many imports on main file
\usepackage{amsmath}
\usepackage{amssymb}
\usepackage{amsthm}
\usepackage{enumerate}
\usepackage{enumitem}
\usepackage{float}
\usepackage{graphicx}
\usepackage{hyperref}
\usepackage{ifthen}
\usepackage{lipsum}
\usepackage{mdframed}
\usepackage{tcolorbox}
\usepackage{xcolor}
\usepackage[sfdefault]{plex-sans}
\usepackage{tikz}

\tcbuselibrary{breakable}

\newcommand{\R}{\ensuremath{\mathbb{R}}}
\newcommand{\N}{\ensuremath{\mathbb{N}\hspace{1.0mm}}}
\newcommand{\Z}{\ensuremath{\mathbb{Z}}}
\newcommand{\Q}{\ensuremath{\mathbb{Q}}}
\newcommand{\C}{\ensuremath{\mathbb{C}}}

% formatting
\newcommand{\hs}{\hspace{1.0mm}} % horizontal space; mostly in equations
\makeatletter
\def\exam@MakeFramed#1{\begin{mdframed}}
\def\endexam@MakeFramed{\end{mdframed}}
\makeatother


\newtcolorbox{customsolutionbox}[1][]{
    colback=white,
    colframe=gray!50!black,
    sharp corners=southwest,
    rounded corners=southeast,
    title=Solution,
    breakable,
    #1 % answer
}

% \newcommand{\customequation}[1]{
%     \vspace{5.0mm}
%     \begin{center}
%         \begin{align*}
%             #1
%         \end{align*}
%     \end{center}
%     % \vspace{5.0mm}
% }

%% credit statement
\newcommand{\creditstatement}[1][alone]{
    % \ifthenelse{\equal{#1}{alone}}{
    \begin{center}
        {\fontsize{14}{16}\selectfont \textbf{\textit{Credit statement: I worked alone on this assignment with reference to: #1} \\}}
    \end{center}
    % }{
    % \begin{center}
    %     {\fontsize{14}{16}\selectfont\textbf{\textit{Credit statement: I discussed the problems with #1, but worked alone on my solutions} \\}}
    % \end{center}
    % }
}

%% heading
\newcommand{\customheading}[4]{ % #1=Date, #2=Title
    \fbox{%
        \begin{minipage}{\textwidth}
        \begin{tikzpicture}[overlay, remember picture]
            \node[anchor=north west, font=\bfseries\itshape] at (0,0) {Abdibaset Bare};
            \node[anchor=north east, font=\bfseries] at (\textwidth,0) {#1};
            \node[anchor=north, font=\Large\bfseries\itshape] at (\textwidth/2,-1.1) {#2};
            \node[anchor=south west, font=\itshape] at (0,-3) {#3};
            \node[anchor=south east, font=\itshape] at (\textwidth,-3) {#4};
        \end{tikzpicture}
        \vspace{3cm}
        \end{minipage}%
    }
}

\newcommand{\customfigure}[3]{
    \begin{figure}[H]
        \centering
        \includegraphics[width=25mm]{#1}
        \caption{#2}
        \label{fig:#3}
    \end{figure}
}

%% definitions
\newtheorem{theorem}{Theorem}
\newtheorem*{definition}{Definition}
\begin{document}
\customheading{\today}{Problem 14}{Fall 2024}{CS30 - Discrete Maths}

\section*{}
\begin{parts}
    \part Suppose a finite number of players play a round-robin tournament, with everyone playing everyone else exactly once. Each match has a winner and a loser (no ties). Prove that we can re-name the players $p_1, \ldots, p_n$ such that $p_1$ has beaten $p_2, p_2$ has beaten $p_3$, and so on. More precisely, for $1 \leq i \leq n-1$, $p_i$ has beaten $p_{i+1}$.

    \begin{customsolutionbox}
        \begin{proof}
            Let $P(n)$ be the predicate that is true if ``for any tournament there are $n$ player, there exists  renaming such that $p_1$ beats $p_2$, $p_2$ beats $p_3$ and so on". We want to prove $P(n)$ is true $\forall n \in \N: P(n)$ \\

            For the base case validation $n = 1$ is vacuously true because it is only one player. If $n = 2$, one of the players wins so can say $p_1$ has beaten $p_2$ without loss of generality, hence this establishes the base case. \\

            Inductive case: Let's fix $k \geq 1$, and the inductive hypothesis is that $P(a)$ is true for $1 \leq a \leq k$. Now we need to show $P(k+1)$ is true, i.e. $P(k+1)$ has been beaten by some player $p_a$ who was beaten by player $p_{a-1}$. Now let's consider the outcome of the tournament between $P(K+1)$ and $P(k)$.
            \begin{itemize}
                \item case 1: $p_{k}$ beats $p_{k+1}$. For this case $P(k+1)$ is true.
                \item Case 2: $p_{k+1}$ beats $p_k$. This breaks the sequence $p_i$ beats $p_{i+1}$. We can, therefore, split the players into two groups, that's, Let $A \subseteq \{p_1, p_2, \ldots, p_{m}\}$ be all players beaten by $p_{k+1}$, and $B$ be the remaining set of players who have beaten $p_{k+1}$. \\

                Now let's consider cases where $p_{k+1}$ joins either $A$ or $B$. Since $|A| \leq k$ and $|B| \leq k$ by inductive hypothesis, we can rename players and re-order such that there exists a group players where $p_{1}$ beats $p_{2}$ who beats $p_{3}$ and so on. Therefore, if $p_{k+1}$ joins $A$ then $p_{k+1}$ can appear before $p_1$ to maintain the order because $p_1$ lost to $p_{k+1}$. Conversely, if they join $B$, we can similarly rename and re-order players as $\{p_1, p_2, \ldots, p_n\}$ such that $p_1$ beats $p_2$, $p_2$ beats $p_3$ and so on, and $p_{k+1}$ appears after $p_n$ to maintain the order because $p_{k+1}$ lost to $p_n$. \\
            \end{itemize}
        \end{proof}
    \end{customsolutionbox}
\end{parts}
\end{document}