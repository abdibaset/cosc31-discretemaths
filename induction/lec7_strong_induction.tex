\documentclass[11pt]{article}

%% import packages to avoid too many imports on main file
\usepackage{amsmath}
\usepackage{amssymb}
\usepackage{amsthm}
\usepackage{enumerate}
\usepackage{enumitem}
\usepackage{float}
\usepackage{graphicx}
\usepackage{hyperref}
\usepackage{ifthen}
\usepackage{lipsum}
\usepackage{mdframed}
\usepackage{tcolorbox}
\usepackage{xcolor}
\usepackage[sfdefault]{plex-sans}
\usepackage{tikz}

\tcbuselibrary{breakable}

\newcommand{\R}{\ensuremath{\mathbb{R}}}
\newcommand{\N}{\ensuremath{\mathbb{N}\hspace{1.0mm}}}
\newcommand{\Z}{\ensuremath{\mathbb{Z}}}
\newcommand{\Q}{\ensuremath{\mathbb{Q}}}
\newcommand{\C}{\ensuremath{\mathbb{C}}}

% formatting
\newcommand{\hs}{\hspace{1.0mm}} % horizontal space; mostly in equations
\makeatletter
\def\exam@MakeFramed#1{\begin{mdframed}}
\def\endexam@MakeFramed{\end{mdframed}}
\makeatother


\newtcolorbox{customsolutionbox}[1][]{
    colback=white,
    colframe=gray!50!black,
    sharp corners=southwest,
    rounded corners=southeast,
    title=Solution,
    breakable,
    #1 % answer
}

% \newcommand{\customequation}[1]{
%     \vspace{5.0mm}
%     \begin{center}
%         \begin{align*}
%             #1
%         \end{align*}
%     \end{center}
%     % \vspace{5.0mm}
% }

%% credit statement
\newcommand{\creditstatement}[1][alone]{
    % \ifthenelse{\equal{#1}{alone}}{
    \begin{center}
        {\fontsize{14}{16}\selectfont \textbf{\textit{Credit statement: I worked alone on this assignment with reference to: #1} \\}}
    \end{center}
    % }{
    % \begin{center}
    %     {\fontsize{14}{16}\selectfont\textbf{\textit{Credit statement: I discussed the problems with #1, but worked alone on my solutions} \\}}
    % \end{center}
    % }
}

%% heading
\newcommand{\customheading}[4]{ % #1=Date, #2=Title
    \fbox{%
        \begin{minipage}{\textwidth}
        \begin{tikzpicture}[overlay, remember picture]
            \node[anchor=north west, font=\bfseries\itshape] at (0,0) {Abdibaset Bare};
            \node[anchor=north east, font=\bfseries] at (\textwidth,0) {#1};
            \node[anchor=north, font=\Large\bfseries\itshape] at (\textwidth/2,-1.1) {#2};
            \node[anchor=south west, font=\itshape] at (0,-3) {#3};
            \node[anchor=south east, font=\itshape] at (\textwidth,-3) {#4};
        \end{tikzpicture}
        \vspace{3cm}
        \end{minipage}%
    }
}

\newcommand{\customfigure}[3]{
    \begin{figure}[H]
        \centering
        \includegraphics[width=25mm]{#1}
        \caption{#2}
        \label{fig:#3}
    \end{figure}
}

%% definitions
\newtheorem{theorem}{Theorem}
\newtheorem*{definition}{Definition}
\begin{document}
\customheading{\today}{Strong Induction}{Fall 2024}{CS30 - Discrete Maths}

\section*{}
\begin{definition}
    Strong Induction: We assume the value holds for all values below$k$
\end{definition}

\begin{theorem}
    Every natural number $\geq 2$ can be written as a product of its primes and $1$
\end{theorem}
\begin{proof}
    In predicate logic: $p_i$ is the $i^{th}$ prime number \\
    Let $P(p_1 \cdot p_2 \cdot \ldots \cdot p_i)$ the product of prime numbers
    \begin{gather}
        \forall n \in \N, n \geq 2: n = 1 \cdot p_1 \cdot p_2 \cdot \ldots \cdot p_i \\
        \forall n \in \N, n \geq 2: n = P(p_1 \cdot p_2 \cdot \ldots \cdot p_i)
    \end{gather}
    Base case
    \begin{gather}
        n = 2; n = 1 \cdot 2
    \end{gather}

    Inductive case: Fix $k \geq 2$ \\
    Inductive hypothesis. $P(l)$ is true for $2 \leq l \leq k$ \\
    Goal: $k+1$. Here we can consider two cases, i.e., when $k+1$ is prime, and when $k+1$ is not prime. \\

    Suppose $k+1$ is prime, then $k+1 = (k+1) \cdot 1$ because a prime number a multiple of itself only and $1$ \\

    If $k+1$ is not prime, then there are prime numbers $a \cdot b$ such that $k+1 = 1 \cdot a \cdot b$ where $2 \leq a \leq k$ and $2 \leq b \leq k$. By inductive hypothesis $P(a)$ and $P(b)$ are true.
\end{proof}

\textbf{The change problem: } In the country of Borduria, they have three types of coins: a cent, a szlapot,
and a dinar. A szlapot is worth 3 cents and a dinar is worth 7 cents. You have an unending supply of szlapots and dinars; show that any amount $\geq 12$ cents can be made with only szlapots and dinars. You may have heard of similar such puzzles. In Math terms, it is stating the following theorem.

\begin{theorem}
    Prove that any number $n \geq 12$ can be expressed as $3x + 7y$
\end{theorem}

\begin{proof}
    For this case $n = 3x + 7y \geq 12$. \\
    In predicate logic
    \begin{gather}
        \forall n \geq 12, n \in \N, \exists x \in \N, \exists y \in \N: n = 3x + 7y
    \end{gather}
    Base case: $x = 4$ and $y = 0$ \\
    Inductive case: Fix $k \geq 12$. \\
    Inductive hypothesis: Assume $P(1), P(2), \ldots, P(l)$ for all $12 \leq l \leq k$ \\
    Goal: Prove that for $k+1$ there exists some $x$ and $y$
    \begin{gather}
        k + 1 = 3x + 7y
    \end{gather}
    Similar to the previous idea of weak induction, we can piggyback to the $k$ value by subtracting either $3$ or $7$. This is assumes that the different between $k$ and $k+1$ is either of those numbers. However, we need to ensure that $n \geq 12$ always. Hence, we need to revise the base case to account for these edge cases. \\

    Suppose, we piggyback by assuming the difference between $k$ and $k+1$ = 3, the base case should be at least $14$, hence \textbf{Inductive hypothesis is } $P(l)$ is true for the domain $14 \leq l \leq k$ \\
    Fix $k \geq 14$
    \begin{gather}
        k + 1 = 3x + 3 + 7y \\
        k + 1 = 3(x + 1) + 7y \\
        x^* = x + 1; y = y
    \end{gather}
    Here $y$ is unchanged. This proves that $P(k+1)$, that's $n$ can be written as $3x + 7y$.

\end{proof}
\end{document}