\documentclass[10pt]{exam}

%% import packages to avoid too many imports on main file
\usepackage{amsmath}
\usepackage{amssymb}
\usepackage{amsthm}
\usepackage{enumerate}
\usepackage{enumitem}
\usepackage{float}
\usepackage{graphicx}
\usepackage{hyperref}
\usepackage{ifthen}
\usepackage{lipsum}
\usepackage{mdframed}
\usepackage{tcolorbox}
\usepackage{xcolor}
\usepackage[sfdefault]{plex-sans}
\usepackage{tikz}

\tcbuselibrary{breakable}

\newcommand{\R}{\ensuremath{\mathbb{R}}}
\newcommand{\N}{\ensuremath{\mathbb{N}\hspace{1.0mm}}}
\newcommand{\Z}{\ensuremath{\mathbb{Z}}}
\newcommand{\Q}{\ensuremath{\mathbb{Q}}}
\newcommand{\C}{\ensuremath{\mathbb{C}}}

% formatting
\newcommand{\hs}{\hspace{1.0mm}} % horizontal space; mostly in equations
\makeatletter
\def\exam@MakeFramed#1{\begin{mdframed}}
\def\endexam@MakeFramed{\end{mdframed}}
\makeatother


\newtcolorbox{customsolutionbox}[1][]{
    colback=white,
    colframe=gray!50!black,
    sharp corners=southwest,
    rounded corners=southeast,
    title=Solution,
    breakable,
    #1 % answer
}

% \newcommand{\customequation}[1]{
%     \vspace{5.0mm}
%     \begin{center}
%         \begin{align*}
%             #1
%         \end{align*}
%     \end{center}
%     % \vspace{5.0mm}
% }

%% credit statement
\newcommand{\creditstatement}[1][alone]{
    % \ifthenelse{\equal{#1}{alone}}{
    \begin{center}
        {\fontsize{14}{16}\selectfont \textbf{\textit{Credit statement: I worked alone on this assignment with reference to: #1} \\}}
    \end{center}
    % }{
    % \begin{center}
    %     {\fontsize{14}{16}\selectfont\textbf{\textit{Credit statement: I discussed the problems with #1, but worked alone on my solutions} \\}}
    % \end{center}
    % }
}

%% heading
\newcommand{\customheading}[4]{ % #1=Date, #2=Title
    \fbox{%
        \begin{minipage}{\textwidth}
        \begin{tikzpicture}[overlay, remember picture]
            \node[anchor=north west, font=\bfseries\itshape] at (0,0) {Abdibaset Bare};
            \node[anchor=north east, font=\bfseries] at (\textwidth,0) {#1};
            \node[anchor=north, font=\Large\bfseries\itshape] at (\textwidth/2,-1.1) {#2};
            \node[anchor=south west, font=\itshape] at (0,-3) {#3};
            \node[anchor=south east, font=\itshape] at (\textwidth,-3) {#4};
        \end{tikzpicture}
        \vspace{3cm}
        \end{minipage}%
    }
}

\newcommand{\customfigure}[3]{
    \begin{figure}[H]
        \centering
        \includegraphics[width=25mm]{#1}
        \caption{#2}
        \label{fig:#3}
    \end{figure}
}

%% definitions
\newtheorem{theorem}{Theorem}
\newtheorem*{definition}{Definition}
\begin{document}
\customheading{\today}{Problem 4: Fibonacci form an additive basis of \N}{Fall 2024}{CS30: Discrete Maths}

\section*{}
\begin{parts}
    \part The Fibonacci numbers are defined as $F_1 = 1, F_2 = 1$ and $F_n = F_{n-1} + F_{n-2}$ for $n \geq 3$. Thus, the first few Fibonacci numbers are $1, 1, 2, 3, 5, 8, 13, 21, \ldots$ \\
    Prove that every natural number can be written as a sum of one or more \textit{distinct} Fibonacci numbers.

    \begin{customsolutionbox}
        \proof Let $P(n)$ be true if any number can be written as a sum of distinct Fibonacci numbers. \\

        Base case $n = 3$:

        \begin{gather}
            n = 1; \text{ Set of Fibonacci numbers }\{1\} \\
            n = 2; \text{ Set of Fibonacci numbers } \{2\} \\
            n = 3; \text{ Set of Fibonacci numbers } \{1, 2\}
        \end{gather}
        The base case $n=1, n=2, n=3$ can be written as a sum of distinct Fibonacci numbers. \\

        Inductive case: Fix $k \geq 3$. The inductive hypothesis is that $P(a)$, for $1 \leq a \leq k$. This implies there exists a set of Fibonacci numbers that add to any natural number $k$. \\

        Goal: Prove that $k+1$ is a sum of distinct Fibonacci numbers. \\

        Let $F_m$ be largest Fibonacci number that's less than or equal $k+1$, i.e.. $k+1 \geq F_m$. Suppose we subtract $F_m$ from $k+1$, then we have:
        \begin{gather}
            (k+1) - F_m \geq 0
        \end{gather}
        Therefore, we can consider two cases below based on the inequality shown above.

        \begin{itemize}
            \item Case 1:  $(k+1)-F_m = 0$. In this case, $k+1$ itself is a Fibonacci number hence $S := \{F_m\}$
            \item Case 2: Suppose $(k+1)- F_m > 0$. Let $a = k + 1 - F_m$ for brevity.
            Since $a \leq k$, by inductive hypothesis, there exists some set of Fibonacci numbers $A$ such that $\sum_{i \in A} i = a$.
            We now claim that $F_m \not \in A$, and therefore the $S := \{F_m\} \cup A$ satisifies $\sum_{x \in  S} x = a + F_m = k+1$, establishing $P(k+1)$. Suppose for the sake of contradiction $F_m \in A$ then $F_m \leq a$. However, $2 \cdot F_m > k+1$ hence the contradiction.
        \end{itemize}
    \end{customsolutionbox}
\end{parts}
\end{document}